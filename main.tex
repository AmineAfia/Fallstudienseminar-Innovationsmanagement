%\documentclass[ngerman]{scrartcl} -kein chapter definition, was wir brauch
\documentclass[11pt, ngerman]{scrreprt}
\usepackage{microtype}
\usepackage[T1]{fontenc}
\usepackage[utf8]{inputenc}
\setlength{\parskip}{\medskipamount}
\setlength{\parindent}{0pt}
\usepackage{babel}
\usepackage{float}
\usepackage{amsmath}
\usepackage{amstext}
\usepackage{varioref}
\usepackage[unicode=true,pdfusetitle,
 bookmarks=true,bookmarksnumbered=false,bookmarksopen=false, breaklinks=false,pdfborder={0 0 0},backref=false,colorlinks=false] {hyperref}
\usepackage{graphicx}
\usepackage{setspace}
\usepackage{cleveref}
\usepackage{icomma}
\usepackage{tabu,booktabs}

% % % % % % % % % % % % % % % % % % % % %
%Schriften - nur eine auskommentieren
	%Latin Modern - Standard
		%\usepackage{lmodern}
	%Times Roman für strenge Dozenten
		%\usepackage{times}
	%Garamond - sieht toll aus, aber eher exotisch
		%\usepackage[cmintegrals,cmbraces]{newtxmath}\usepackage{ebgaramond-maths}\usepackage{helvet}
	%Arial für unsere prof
\usepackage{helvet}\renewcommand{\familydefault}{\sfdefault}

%Hurenkinder, Schusterjungen
\widowpenalties=3 10000 10000 150


% % % % % % % % % % % % % % % % % % % % %
%LITERATUR
%Fußnote _ODER_ Amerikanisch auskommentieren

	%Amerikanische In-Text-Zitierweise
		%\usepackage[backend=biber,style=authoryear,maxcitenames=1]{biblatex}\renewcommand{\cite}{\parencite}

%Deutsches Fußnotensystem
\usepackage[backend=biber,style=authortitle-dw,firstfull,maxcitenames=1]{biblatex}\DeclareFieldFormat{title}{\mkbibemph{#1}}\DeclareFieldFormat{citetitle}{\mkbibemph{#1}}\renewcommand{\cite}{\footcite}

%Ressource
\addbibresource{bibliothek.bib}
% % % % % % % % % % % % % % % % % % % % %
\makeatletter
\begin{document}
		%%Hier alle Daten einfügen
\newcommand{\autorname}{Amine Afia\\ Elham Gharibi \\ Laura Ditus \\ Lisa Zinßer}

\newcommand{\amine}{Amine Afia  (21434)}
\newcommand{\elham}{Elham Gharibi  (21434)}
\newcommand{\laura}{Laura Ditus  (21434)}
\newcommand{\lisa}{Lisa Zinßer  (21434)}

\newcommand{\autormail}{Email des Autors}
\newcommand{\autormatr}{21434\\ 21434 \\ 21434 \\ 21434}
\newcommand{\fak}{Institut für Entrepreneurship, Technologie-Management und Innovation (EnTechnon)}
\newcommand{\lehrstuhl}{Lehrstuhl für Innovations- und TechnologieManagement (iTM)}
\newcommand{\lv}{Fallstudienseminar Innovationsmanagement}
\newcommand{\semester}{Winter Semester 2016/2017}
\newcommand{\lp}{Prof. Dr. Marion A. Weissenberger-Eibl}
\newcommand{\arbtyp}{Die Karosserie im Jahr 2040}
\newcommand{\titel}{Fallstudienseminar Innovationsmanagement}

\title{\titel}
\author{\autorname}
		\begin{titlepage}

%\begin{figure}
    \centering
        \includegraphics[width=5cm]{res/KIT.jpg}
%\end{figure}
\vspace*{2cm}

\begin{tabu} to \textwidth {X[c]}
	\toprule
	\toprule[2pt]
	\huge{\textsc{\titel}}\\ \\
	\bottomrule[2pt]
	\bottomrule
	\\ \Large{\textsc{\arbtyp}}\\
\end{tabu}
\\[3cm]
\centering{
\textsc{\Large \amine\\ \elham\\ \laura\\ \lisa\\}
}
\\[2.5cm]
\begin{tabu} {>{\itshape}X[r] X[1.5]}
Fakultät & \fak\\
Lehrstuhl & \lehrstuhl\\
Lehrveranstaltung & \lv\\
Semester & \semester\\
Professor & \lp \\
Betreuer & Fanny Seus\\
\end{tabu}

\\[1cm]
%\begin{tabu} {>{\itshape}X[r]}
Karlsruhe, den \today \hfill \hspace{-3cm}
%\end{tabu}
\end{titlepage}
\pagenumbering{gobble}

\tableofcontents

\listoffigures
 
\listoftables

\vfill
%\hrule
%\begin{center}\autorname~$\cdot$ \autormatr~$\cdot$ %\href{mailto:\autormail}{\autormail}\end{center}
		\pagebreak{}


\onehalfspacing
\pagenumbering{arabic}
%\input{texes/eiderkl.tex}
\chapter{Einleitung}
    ... \autocite{amine, elham, laura, lisa}

\chapter{Motivation}
    \section{Problem Definition}
    ...
    \section{Methoden Auswahl}
    ...
    \section{Ziele der Arbeit}
    ...

\chapter{Szenariofeld Analyse}
    \section{Bildung von Einflussfaktoren}
        \subsection{Ermittlung von Einflußfaktoren}
        Brainstorming Phase
        \subsection{Ermittlung von Deskriptoren}
        Deskriptoren (das kann auch in Brainstorming Phase rein)
        \subsection{Verteilung von Einflußfaktoren nach STEEP}
        Mindmap
    \section{Schlüsselfaktoren}
        \subsection{Einflussanalyse}
        \subsection{Relevanzanalyse}
    \section{Beschreibung der Einflussfaktoren}

\chapter{Szenario Bildung}
    \section{Projektionsbündelung}
        \subsection{Cross-impact Analyse}
        \subsection{Projektionsbündel-Reduktion}
    \section{Rohszenarien Bildung}
        \subsection{Clusteranalyse}
    \section{Zukunftsraum-Mapping}
        \subsection{Projektionbündel-Mapping}
    \section{Szenario-Beschreinbung}
        \subsection{Szenario 1}
        \subsection{Szenario 2}
        
\chapter{Szenario Transfer}
    \section{Auswirkungsanalyse}
    \section{Eventualplanung}
    \section{Handlungsempfehlungen}
        \subsection{Szenario 1}
        \subsection{Szenario 2}

\chapter{Fazit}


\appendix
\chapter{Anhang}
\label{chap:appendix}

\section{Anhang 1: ...}
\label{anhang:anhang1}
\paragraph{Opener}
\begin{itemize}
    \item Studenten am KIT
    \item Seminar
    \item Wollten Sie kurz fragen: \textbf{Wie schenken Sie?}
\end{itemize}

		%\printbibliography
		
		%\input{texes/eiderkl.tex}
\end{document}
